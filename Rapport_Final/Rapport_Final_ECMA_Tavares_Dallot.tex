\documentclass[11pt,a4paper]{exam}

% For maths symbols
\usepackage{amsmath, amssymb, bm}
\usepackage{wasysym, bigints}
\usepackage{graphicx}
\usepackage{makecell}

%Pour les affichages de graphiques
\usepackage{tikz}
\usepackage{pgfplots}
\pgfplotsset{compat = newest}


% Langue
\usepackage[french]{babel}
\usepackage[T1]{fontenc}

\title{ECMA : Rapport Final}
\author{Antonio Tavares, Julien Dallot}

%Define a length between lines in linear problems
\newlength{\interlinespace}\setlength{\interlinespace}{7mm}


\usepackage{biblatex} %Imports biblatex package
\addbibresource{references.bib} %Import the bibliography file

\begin{document}
\maketitle

je \cite{Bouquin}

\begin{tikzpicture}{H}
	\begin{axis}[width=\textwidth*0.92,
		ylabel={phase max cost},
		xlabel={cluster capacity($k$)},
		title={Max cost dependance of cluster capacity, with $l=2$}]
		\addplot[domain=2:200, 
		samples=100, color=red]{2*x*log2(x)};
		\addlegendentry{$2 k \log_{2}(k)$}
		
		\addplot[
		color=blue,
		mark=square, mark size=1
		]
		coordinates {
			(2,2)(3,6)(4,10)(5,18)(6,22)(7,32)(8,42)(9,50)(10,58)(11,70)(12,78)(13,88)(14,88)(15,102)(17,110)(20,144)(23,170)(25,190)(30,246)(40,378)(50,492)(60,594)(70,756)(80,850)(85,934)(100,1172)(140,1736)(200,2640)
		};
		\addlegendentry{empirical max cost}
	\end{axis}
\end{tikzpicture}

\section{Heuristiques}

Nous avons réalisé des heuristiques exactes pour les sous problèmes (SP1) et (SP2) en les réduisant à des problèmes de sac à dos continues, sur lesquels une heuristique gloutonne simple 

\printbibliography
\end{document}

