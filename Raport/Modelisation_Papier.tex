\documentclass[11pt,a4paper]{exam}

% For maths symbols
\usepackage{amsmath, amssymb, bm}
\usepackage{wasysym, bigints}
\usepackage{graphicx}

% Langue
\usepackage[french]{babel}

\title{Modélisation Papier}
\author{Antonio Tavares, Julien Dallot}

%Define a length between lines in linear problems
\newlength{\interlinespace}\setlength{\interlinespace}{7mm}

\begin{document}
\maketitle

\begin{questions}
\question Voici le problème statique $\left(\mathcal{P}\right)$:
\begin{align*}
	\left(\mathcal{P}\right) \left\{
	\begin{array}{l c c l c c}
		&\min\limits_{x} &\sum\limits_{ij \in A} d_{ij} \: x_{ij}\\
		&\text{s.c.}\\
		% Contrainte 1
		& &\sum\limits_{ij \in A} p_{i} x_{ij} \quad + \quad p_{t} &\leq S & (1) & \text{(limite sur les sommets empruntés)}\\[\interlinespace]
		% Contrainte 2
		& &\sum\limits_{j \in \delta^{+}(i)} x_{ji} \ - \sum\limits_{j \in \delta^{-}(i)} x_{ij} &= 0 \quad \forall i \in V \setminus \{s, t\} & (2) & \text{(conservation du flot)}\\[\interlinespace]
		% Contrainte 3
		& &\sum\limits_{j \in \delta^{-}(s)} x_{sj} &= 1 & (3) &  \text{(un seul arc sortant de $s$)}\\[\interlinespace]
		% Contrainte 4
		& &\sum\limits_{j \in \delta^{+}(t)} x_{jt} &= 1 & (4) &  \text{(un seul arc rentrant dans $t$)}\\
	\end{array}
	\right.
\end{align*}

\question Modélisation du problème robuste $\left(\mathcal{P}_{r}\right)$:

\begin{align*}
	\left(\mathcal{P}_{r}\right) \left\{
	\begin{array}{l c c l c}
		&\min\limits_{x} \max\limits_{\delta^{1}} &\sum\limits_{ij \in A} d_{ij} \: (1 + \delta^{1}_{ij}) \: x_{ij}\\
		&\text{s.c.}\\
		% Contrainte 1
		& &\sum\limits_{ij \in A} p_{i} \: x_{ij} + \max\limits_{\delta^{2}}\left\{\delta^{2}_{t} \hat{p}_{t} + \sum\limits_{ij \in A} \delta^{2}_{i} \: \hat{p}_{i} \: x_{ij} \right\} &\leq S - p_{t} & (1_{robuste})\\[\interlinespace]
		% Contrainte 2
		& &\sum\limits_{j \in \delta^{+}(i)} x_{ji} \ - \sum\limits_{j \in \delta^{-}(i)} x_{ij} &= 0 \quad \forall i \in V \setminus \{s, t\} & (2_{robuste})\\[\interlinespace]
		% Contrainte 3
		& &\sum\limits_{j \in \delta^{-}(s)} x_{sj} &= 1 & (3_{robuste})\\[\interlinespace]
		% Contrainte 4
		& &\sum\limits_{j \in \delta^{+}(t)} x_{jt} &= 1 & (4_{robuste})\\
	\end{array}
	\right.
\end{align*}

\question Résolution par plans coupants et LazyCallback. On réexprime d'abord le problème robuste avec les ensembles $\mathcal{U}^{1*}$ et $\mathcal{U}^{2*}$.

\begin{align*}
\left(\mathcal{P}_{r}\right) \left\{
\begin{array}{l c c l c}
&\min\limits_{z} &z\\
&\text{s.c.}\\
% Contrainte 0
& &\sum\limits_{ij \in A} d^{1}_{ij} \: x_{ij} &\leq z  \quad \text{$\forall d^{1} \in \mathcal{U}^{1*}$} & (0)\\[\interlinespace]
% Contrainte 1
& &\sum\limits_{ij \in A} p^{2}_{i} x_{ij} \quad + \quad p^{2}_{t} &\leq S \quad \text{$\forall p^{2} \in \mathcal{U}^{2*}$} & (1_{robuste})\\[\interlinespace]
% Contrainte 2
& &\sum\limits_{j \in \delta^{+}(i)} x_{ji} \ - \sum\limits_{j \in \delta^{-}(i)} x_{ij} &= 0 \quad \forall i \in V \setminus \{s, t\} & (2_{robuste})\\[\interlinespace]
% Contrainte 3
& &\sum\limits_{j \in \delta^{-}(s)} x_{sj} &= 1 & (3_{robuste})\\[\interlinespace]
% Contrainte 4
& &\sum\limits_{j \in \delta^{+}(t)} x_{jt} &= 1 & (4_{robuste})\\
\end{array}
\right.
\end{align*}

(Je sais pas encore quels ensembles $\mathcal{U}^{1*}$ et $\mathcal{U}^{2*}$ choisir au début, deux sous ensembles aléatoires et petits de $\mathcal{U}^{1}$ et $\mathcal{U}^{2}$ ?)\\

Il y a deux sous problèmes à résoudre, $(SP_{0})$ et $(SP_{1})$:


\begin{align*}
&\left(SP_{0}\right) \left\{
\begin{array}{ll}
 \max\limits_{\delta^{1}} &\sum\limits_{ij \in A} d_{ij} \ (1 + \delta^{1}_{ij}) \ x^{*}_{ij}\\
 \text{s.c}\\
 &\sum\limits_{ij \in A} \delta^{1}_{ij} \leq d_{1} \quad \forall ij \in A\\[\interlinespace]
 &\delta^{1}_{ij} \leq D_{ij} \quad \forall ij \in A
 \end{array}
 \right.
&&\left(SP_{1}\right) \left\{
\begin{array}{l l}
 \max\limits_{\delta^{2}} &\sum\limits_{i \in V} (p_{i} + \delta^{2}_{i} \hat{p}_{i}) \ x^{*}_{ij} \quad + \quad p_{t} + \delta^{2}_{t} \hat{p}_{t}\\
\text{s.c}\\
&\sum\limits_{i \in V} \delta^{2}_{i} \leq d_{2} \quad \forall i \in V\\[\interlinespace]
&\delta^{2}_{i} \leq 2 \quad \forall i \in V
\end{array}
\right.
\end{align*}

avec $x^{*}$ une solution courante du problème maître.

\end{questions}

\end{document}

